\documentclass[twocolumn]{article}

\usepackage{amsmath}
\usepackage{amssymb}
\usepackage{enumitem}

\setlist{nosep}

\newcommand{\qed}{$\blacksquare$}
\newcommand{\br}{\vspace{\baselineskip}}
\newcommand{\COMPLEMENT}{^\mathsf{c}}
\let\complement\COMPLEMENT
\let\emptyset\varnothing
\let\eps\varepsilon

\newcommand{\naturals}{\mathbb{N}}
\newcommand{\integers}{\mathbb{Z}}
\newcommand{\rationals}{\mathbb{Q}}
\newcommand{\irrationals}{\rationals\complement}
\newcommand{\reals}{\mathbb{R}}
\newcommand{\complex}{\mathbb{C}}

\setlength{\parindent}{0pt}

\usepackage{changepage}
\newcommand{\boxthis}[1]{\fbox{\begin{minipage}{\linewidth}{#1}\end{minipage}}}

\usepackage{etoolbox}
\newcommand{\zerodisplayskips}{%
  \setlength{\abovedisplayskip}{0pt}%
  \setlength{\belowdisplayskip}{0pt}%
  \setlength{\abovedisplayshortskip}{0pt}%
  \setlength{\belowdisplayshortskip}{0pt}}
\appto{\normalsize}{\zerodisplayskips}
\appto{\small}{\zerodisplayskips}
\appto{\footnotesize}{\zerodisplayskips}

\begin{document}

Recall. Let $f : A \rightarrow B$. Then $f$
\begin{enumerate}
	\item
		is said to be injective if $f(a_1) = f(a_2)$ implies $a_1 = a_2$ for all $a_1, a_2 \in A$.
	\item
		is said to be surjective if for all $b \in B$, there exists $a \in A$ such that $f(a) = b$.
	\item
		is said to be bijective if it is both injective and surjective.
	\item
		has a left inverse if there exists $g : B \rightarrow A$ such that $g $
\end{enumerate}


\end{document}

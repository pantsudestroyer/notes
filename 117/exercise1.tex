\documentclass{article}

\usepackage{amsmath}
\usepackage{amssymb}

\newcommand{\naturals}{\mathbb{N}}
\newcommand{\integers}{\mathbb{Z}}
\newcommand{\rationals}{\mathbb{Q}}
\newcommand{\reals}{\mathbb{R}}
\newcommand{\complex}{\mathbb{C}}

\begin{document}

\begin{enumerate}
	\item
		Show that 1001 is divisible by 7, by 11, and by 13.
	\item
		Find the quotient and remainder in the division algorithm when the following numbers are divided by 17.
		\begin{enumerate}
			\item
				100
			\item
				289
			\item
				-100
		\end{enumerate}
	\item
		Prove the following:
		\begin{enumerate}
			\item
				If $a|b$ and $b|a$, then $a = \pm b$.
			\item
				If $a|b$ and $c|d$, then $ac | bd$.
			\item
				If $a|b$, then $a^k | b^k$ for all positive integers $k$.
		\end{enumerate}
	\item
		Show that if $a$ is an integer, then 3 divides $a^3 - a$.
	\item
		Show that the fourth power of any integer is either of the form $5k$ or $5k + 1$.
	\item
		Verify that $2 | a(a + 1)$ and $3 | a(a + 1)(a + 2)$ for any integer $a$.
	\item
		Show that if $a$ is odd, then $32 | (a^2 + 3)(a^2 + 7)$.
	\item
		Prove that if $a$ is an integer not divisible by 2 or 3, then $24 | (a^2 + 23)$.
	\item
		Prove that the sum of the squares of two odd integers cannot be a perfect square.
\end{enumerate}

\pagebreak

Solutions.
\begin{enumerate}
	\item
		Note that $1001 = 7 \cdot 143 = 11 \cdot 91 = 13 \cdot 77$ where $143, 91, 77 \in \integers$ so that 1001 is divisible by 7, by 11, and by 13.
	\item
		\begin{enumerate}
			\item
				$100 = 17 \cdot 5 + 15$ so that the quotient and remainder when 100 is divided by 17 are 5 and 15, respectively.
			\item
				$289 = 17 \cdot 17 + 0$ so that the quotient and remainder when 289 is divided by 17 are 17 and 0, respectively.
			\item
				$-100 = 17 \cdot -6 + 2$ so that the quotient and remainder when $-100$ is divided by 17 are $-6$ and 2, respectively.
		\end{enumerate}
	\item
		\begin{enumerate}
			\item
				Suppose that $a|b$ and $b|a$. Then, by definition, there exists $m, n \in \integers$ such that $b = am$ and $a = bn$. If $b = 0$, then $a = 0n = 0$. Note that $0 = \pm 0$ so that $a = \pm b$. Otherwise, assume that $b \neq 0$. Then, $b = (bn)m = b(nm)$ so that $1 = nm$. Since $m \in \integers$, then $n | 1$ so that $n = \pm 1$. Thus, $a = b \cdot \pm 1 = \pm b$.
			\item
				Suppose that $a|b$ and $c|d$. Then, by definition, there exists $m, n \in \integers$ such that $b = am$ and $d = cn$. Hence, $bd = (am)(cn) = ac(mn)$ where $mn \in \integers$. Thus, by definition, $ac | bd$.
			\item
				Suppose that $a|b$. Then, by definition, there exists $m \in \integers$ such that $b = am$. Hence, $b^1 = a^1m$ so that $a^1 | b^1$. Let $k$ be a positive integer. Suppose that $a^k | b^k$. Then, by definition, there exists $n \in \integers$ such that $b^k = a^kn$. Thus, $(b^k)(b) = (a^kn)(am)$. Hence, $b^{k + 1} = a^{k + 1}(mn)$ where $mn \in \integers$. Thus, by definition, $a^{k + 1} | b^{k + 1}$. By the principle of mathematical induction, $a^k | b^k$ for all positive integers $k$.
		\end{enumerate}
	\item
		By the division algorithm, any integer $a$ can be written in one of the following forms:
		$$a = 3q\text{,}$$
		$$a = 3q + 1\text{, and}$$
		$$a = 3q + 2$$
		where $q \in \integers$. If $a = 3q$ then $a^3 - a = 27q^3 - 3q = 3(9q^3 - q)$, where $9q^3 - q \in \integers$ so that $3 | a^3 - a$. If $a = 3q + 1$ then $a^3 - a = 27q^3 + 27q^2 + 9q + 1 - (3q + 1) = 27q^3 + 27q^2 + 6q = 3(9q^3 + 9q^2 + 2q)$, where $9q^3 + 9q^2 + 2 \in \integers$ so that $3 | a^3 - a$. If $a = 3q + 2$ then $a^3 - a = 27q^3 + 54q^2 + 36q + 8 - (3q + 2) = 27q^3 + 54q^2 + 33q + 6 = 3(9q^3 + 18q^2 + 11q + 2)$, where $9q^3 + 18q^2 + 11q + 2 \in \integers$ so that $3 | a^3 - a$. Thus, $3 | a^3 - a$ for any integer $a$.
	\item
		Let $n$ be an integer. By the division algorithm, $n$ can be written in one of the following forms:
		$$n = 5q\text{,}$$
		$$n = 5q + 1\text{,}$$
		$$n = 5q + 2\text{,}$$
		$$n = 5q + 3\text{, and}$$
		$$n = 5q + 4$$
		where $q \in \integers$. If $n = 5q$ then $n^4 = 625q^4 = 5(125q^4)$. Taking $k = 125q^4 \in \integers$, we can write $n^4$ as $5k$. If $n = 5q + 1$ then $n^4 = 625q^4 + 500q^3 + 150q^2 + 20q + 1 = 5(125q^4 + 100q^3 + 30q^2 + 4q) + 1$. Taking $k = 125q^4 + 100q^3 + 30q^2 + 4q \in \integers$, we can write $n^4$ as $5k + 1$. If $n = 5q + 2$ then $n^4 = 625q^4 + 1000q^3 + 600q^2 + 160q + 16 = 5(125q^4 + 200q^3 + 120q^2 + 32q + 3) + 1$. Taking $k = 125q^4 + 200q^3 + 120q^2 + 32q + 3 \in \integers$, we can write $n^4$ as $5k + 1$. If $n = 5q + 3$ then $n^4 = 625q^4 + 1500q^3 + 1350q^2 + 540q + 81) = 5(125q^4 + 300q^3 + 270q^2 + 108q + 16) + 1$. Taking $k = 125q^4 + 300q^3 + 270q^2 + 108q + 16 \in \integers$, we can write $n^4$ as $5k + 1$. If $n = 5q + 4$ then $n^4 = 625q^4 + 2000q^3 + 2400q^2 + 1280q + 256 = 5(125q^4 + 400q^3 + 480q^2 + 256q + 51) + 1$. Taking $k = 125q^4 + 400q^3 + 480q^2 + 256q + 51) \in \integers$, we can write $n^4$ as $5k + 1$. Thus, the fourth power of any integer is either of the form $5k$ or $5k + 1$.
	\item
		Let $a$ be an integer. Then $a(a + 1) \in \integers$ so that $3 | 3a(a + 1)$. We know that $3 | a^3 - a = a(a + 1)(a - 1)$. Thus, $3 | a(a + 1)(a - 1) + 3a(a + 1) = a(a + 1)(a - 1 + 3) = a(a + 1)(a + 2)$. By the division algorithm, either $a = 2q$ or $a = 2q + 1$ for some $q \in \integers$. If $a = 2q$ then $a(a + 1) = 2q(2q + 1) = 2(q(2q + 1))$, where $q(2q + 1) \in \integers$ so that $2 | a(a + 1)$. If $a = 2q + 1$ then $a(a + 1) = (2q + 1)(2q + 1 + 1) = 2((q + 1)(2q + 1))$, where $(q + 1)(2q + 1) \in \integers$ so that $2 | a(a + 1)$.
	\item
		Suppose that $a$ is odd. Thus, $a = 2q + 1$ for some $q \in \integers$. Hence, $a^2 + 3 = 4q^2 + 4q + 1 + 3 = 4(q^2 + q + 1)$ and $a^2 + 7 = 4q^2 + 4q + 1 + 7 = 4(q^2 + q + 2)$. We know that $2 | q(q + 1)$ and $2 | 2$ so that $2 | q(q + 1) + 2 = q^2 + q + 2$. Thus, by definition, $q^2 + q + 2 = 2k$ for some $k \in \integers$. Hence, $(a^2 + 3)(a^2 + 7) = (4(q^2 + q + 1))(4(2k)) = 32k(q^2 + q + 1) = 32(k(q^2 + q + 1))$, where $k(q^2 + q + 1) \in \integers$ so that $32 | (a^2 + 3)(a^2 + 7)$.
	\item
		Suppose that $a$ is an integer not divisible by 2 or 3. By the division algorithm, any integer $a$ can be written in one of the following forms:
		$$a = 6q\text{,}$$
		$$a = 6q + 1\text{,}$$
		$$a = 6q + 2\text{,}$$
		$$a = 6q + 3\text{,}$$
		$$a = 6q + 4\text{, and}$$
		$$a = 6q + 5$$
		where $q \in \integers$. Since $a$ is not divisible by 3, we get that $a \neq 6q = 3(2q)$ and $a \neq 6q + 3 = 3(2q + 1)$. Since $a$ is not divisible by 2, we additionally get that $a \neq 6q + 2 = 2(3q + 1)$ and $a \neq 6q + 4 = 2(3q + 2)$. Thus, either $a = 6q + 1$ or $a = 6q + 5$. If $a = 6q + 1$ then $a^2 + 23 = 36q^2 + 12q + 1 + 23 = 24(q^2 + 1) + 12(q^2 + q)$. We know that $2 | q^2 + q$ so that for some $k \in \integers$, $q^2 + q = 2k$. Thus, $a^2 + 23 = 24(q^2 + 1) + 12(2k) = 24(q^2 + 1 + k)$, where $q^2 + 1 + k \in \integers$ so that $24 | a^2 + 23$. If $a = 6q + 5$ then $a^2 + 23 = 36q^2 + 60q + 25 + 23 = 24(q^2 + 2q + 2) + 12(q^2 + q) = 24(q^2 + 2q + 2) + 12(2k) = 24(q^2 + 2q + 2 + k)$, where $q^2 + 2q + 2 + k \in \integers$ so that $24 | a^2 + 23$.
	\item
		Let $a$ and $b$ be odd integers. Thus, $a = 2m + 1$ and $b = 2n + 1$ for some $m, n \in \integers$. Assume, for the sake of contradiction, that the sum of their squares is a perfect square. Thus, for some $c \in \integers$, $a^2 + b^2 = c^2$. Note that $a^2 = 4m^2 + 4m + 1$ and $b^2 = 4n^2 + 4n + 1$ so that $c^2 = 4m^2 + 4m + 1 + 4n^2 + 4n + 1 = 4(m^2 + m + n^2 + n) + 2$, where $m^2 + m + n^2 + n \in \integers$ and $0 \leq 2 < 4$ so that the remainder when $c^2$ is divided by 4 is 2. By the division algorithm, either $c = 2k$ or $c = 2k + 1$ for some $k \in \integers$. Thus, either $c^2 = 4k^2$ or $c^2 = 4k^2 + 4k + 1 = 4(k^2 + k) + 1$, where $k^2, k^2 + k \in \integers$ and $0 \leq 0, 1 < 4$ so that the remainder when $c^2$ is divided by 4 is either 0 or 1. Thus, we get that 2 is either 0 or 1, which is absurd. Thus, the sum of the squares of two odd integers cannot be a perfect square.
\end{enumerate}

\end{document}

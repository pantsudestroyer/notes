\documentclass[twocolumn]{article}

\usepackage[margin=1in]{geometry}

\usepackage{amsmath}
\usepackage{amssymb}
\usepackage{enumitem}

\setlist{nosep}

\newcommand{\qed}{$\blacksquare$}
\newcommand{\br}{\vspace{\baselineskip}}
\newcommand{\COMPLEMENT}{^\mathsf{c}}
\let\complement\COMPLEMENT
\let\emptyset\varnothing
\let\eps\varepsilon

\newcommand{\naturals}{\mathbb{N}}
\newcommand{\integers}{\mathbb{Z}}
\newcommand{\rationals}{\mathbb{Q}}
\newcommand{\irrationals}{\rationals\complement}
\newcommand{\reals}{\mathbb{R}}
\newcommand{\complex}{\mathbb{C}}

\setlength{\parindent}{0pt}

\usepackage{changepage}
\newcommand{\boxthis}[1]{\fbox{\begin{minipage}{\linewidth}{#1}\end{minipage}}}

\usepackage{etoolbox}
\newcommand{\zerodisplayskips}{%
  \setlength{\abovedisplayskip}{0pt}%
  \setlength{\belowdisplayskip}{0pt}%
  \setlength{\abovedisplayshortskip}{0pt}%
  \setlength{\belowdisplayshortskip}{0pt}}
\appto{\normalsize}{\zerodisplayskips}
\appto{\small}{\zerodisplayskips}
\appto{\footnotesize}{\zerodisplayskips}

\begin{document}

\pagenumbering{gobble}

Math 123.1 \\
Homework 1 \\
Kirk Bamba

\begin{enumerate}
	\item
	Let $u \in \reals$ and $S \subseteq \reals$ such that $S \neq \emptyset$. Suppose that for every $n \in \naturals$, $u + \frac{1}{n}$ is an upper bound of $S$ and $u - \frac{1}{n}$ is an upper bound of $S$. \\
	Suppose that $u$ is not an upper bound of $S$. Then $u < x$ for some $x \in S$. Let $\eps = x - u > 0$. Then, by the archimedean property, there exists $n \in \naturals$ such that $\frac{1}{n} < \eps$. So $u + \frac{1}{n} < u + \eps = x$, which is absurd since $u + \frac{1}{n}$ is an upper bound of $S$. Thus, $u$ is an upper bound of $S$. \\
	Suppose that $u$ is not the supremum of $S$. Then, there exists some upper bound $u'$ of $S$ such that $u' < u$. Let $\eps' = u - u' > 0$. Then by the archimedean property, there exists $n' \in \naturals$ such that $\frac{1}{n'} < \eps'$. Hence, $\frac{1}{n'} < u - u'$ so that $u' < u - \frac{1}{n'}$. But since $u - \frac{1}{n'}$ is not an upper bound of $S$, there exists $x' \in S$ such that $u - \frac{1}{n'} < x'$. Then $u' < x'$, which contradicts our assumption that $u'$ is an upper bound of $S$. Therefore, $u$ is the supremum of $S$, as required.
	
	\item
	Lemma 1. \\
	Let $R,\, S$ be sets. Suppose that $R$ is closed and $S \subseteq R$. Then $\overline{S} \subseteq R$. \\
	Proof. \\
	Let $R,\, S$ be sets. Suppose that $R$ is closed and $S \subseteq R$. Let
	$$T := \{ U \mid S \subseteq U \text{, } U \text{ is closed} \} \text{.}$$
	Then $\overline{S} = \bigcap T$ and $R \in T$. Hence, $\overline{S} \cap R = \overline{S}$ so that $\overline{S} \subseteq R$, as required. \\
	Lemma 2. \\
	$S \subseteq \overline{S}$ for any set $S$. \\
	Proof. \\
	Let $S$ be a set. Let
	$$T := \{ U \mid S \subseteq U \text{, } U \text{ is closed} \} \text{.}$$
	Suppose that $S \not\subseteq \overline{S}$. Then for some $x \in S$, $x \not\in \overline{S}$. Since $S = \bigcap T$, then there exists $X \in T$ such that $x \not\in X$. But $S \subseteq X$ and $x \in S$ so that $x \in X$. Clearly, this is a contradiction. Hence, $S \subseteq \overline{S}$, as required. \\
	
	Let $A,\, B$ be sets. 
	by Lemma 2, we have that $A \subseteq \overline{A}$ and $B \subseteq \overline{B}$. Hence, $A \cup B \subseteq \overline{A} \cup \overline{B}$. Recall that the finite union and arbitrary intersection of closed sets are closed sets. Hence, both $\overline{A} = \bigcap \{ U \mid A \subseteq U \text{, } U \text{ is closed} \}$ and $\overline{B} = \bigcap \{ U \mid B \subseteq U \text{, } U \text{ is closed} \}$ are closed. So $\overline{A} \cup \overline{B}$ is a union of two closed sets and is thus also closed. Hence, by Lemma 1, $\overline{A \cup B} \subseteq \overline{A} \cup \overline{B}$. \\
	Note that $\overline{A \cup B} = \bigcap \{ U \mid A \cup B \subseteq U \text{, } U \text{ is closed} \}$ is also an intersection of closed sets and is thus closed. Also, by Lemma 2, $A \cup B \subseteq \overline{A \cup B}$. But $A \subseteq A \cup B$ and $B \subseteq A \cup B$ so that $A \subseteq \overline{A \cup B}$ and $B \subseteq \overline{A \cup B}$. Hence, by Lemma 1, $\overline{A} \subseteq \overline{A \cup B}$ and $\overline{B} \subseteq \overline{A \cup B}$. Thus, $\overline{A} \cup \overline{B} \subseteq \overline{A \cup B}$. \\
	Therefore, $\overline{A \cup B} = \overline{A} \cup \overline{B}$, as required.
	
	\item
	\begin{enumerate}
		\item
		Let $n \in \naturals$. So $4x_n (1 - x_{n + 1}) > 1$ and $0 < x_n$. Hence, $1 - x_{n + 1} > \frac{1}{4x_n}$ so that $x_{n + 1} < 1 - \frac{1}{4x_n}$. \\
		\boxthis{
		Claim. \\
		$1 - \frac{1}{4y} \leq y$ for any $y > 0$. \\
		Proof. \\
		Let $y > 0$. So $\frac{1}{2y} - 1 \in \reals$ so that by the trivial inequality, $(\frac{1}{2y} - 1)^2 \geq 0$. Hence, $\frac{1}{4y^2} - \frac{1}{y} + 1 \geq 0$. So $1 \geq \frac{1}{y} - \frac{1}{4y^2}$. Multiplying by $y$, we get $y \geq 1 - \frac{1}{4y}$, as required.
		}
		In particular, $1 - \frac{1}{4x_n} \leq x_n$ so that $x_{n + 1} < x_n$. Thus, $(x_n)$ is decreasing. Therefore, $(x_n)$ is monotone, as required.
		\item
		Note that $(x_n)$ is bounded above by 1 and bounded below by 0. Hence, $(x_n)$ is bounded. So by the monotone convergence theorem, $(x_n)$ converges. Then there exists $x \in \reals$ such that $x = \lim x_n$. Then by a previous theorem, $x = \lim x_{n + 1}$. Consider the sequence
		\begin{align*}
		(z_n)  := &\, (4x_n (1 + (-1) \cdot x_{n + 1})) \\
				= &\, (4x_n (1 - x_{n + 1})) \text{.}
		\end{align*}
		Note that $(4)$, $(x_n)$, $(1)$, $(-1)$, and $(x_{n + 1})$ are all convergent sequences so that by a previous theorem, $(z_n)$ converges and 
		\begin{align*}
		\lim z_n 	= &\, 4 (\lim x_n) (1 + (-1) \cdot (\lim x_{n + 1}) \\
					= &\, 4x (1 + (-1) \cdot x) \\
					= &\, 4x - 4x^2 \text{.}
		\end{align*} But $z_n = 4x_n (1 - x_{n + 1}) > 1$ for all $n \in \naturals$ so that $\lim z_n \geq \lim 1$. Hence, $4x - 4x^2 \geq 1$. So $0 \geq 4x^2 - 4x + 1 = (2x - 1)^2$. But $2x - 1 \in \reals$ so that by the trivial inequality, $0 \leq (2x - 1)^2$. Thus, $(2x - 1)^2 = 0$. Hence, $2x - 1 = 0$ so that $x = \frac{1}{2}$. Therefore, the limit of $(x_n)$ is $\frac{1}{2}$.
	\end{enumerate}
\end{enumerate}

\end{document}

\documentclass{article}

\usepackage{amsmath}
\usepackage{amssymb}

\newcommand{\reals}{\mathbb{R}}
\let\emptyset\varnothing

\begin{document}

Advanced Calculus I
\begin{itemize}
	\item
		single variable real analysis
\end{itemize}

Real numbers ($\reals$)
\begin{itemize}
	\item
		$(\reals, +, \cdot)$ is a field.
	\item
		$\reals$ with $\geq$ is a partially ordered set.
\end{itemize}

Absolute value
\begin{itemize}
	\item
		$|x| = \begin{cases}
			x & x \geq 0 \\
			-x & x < 0 
		\end{cases}$
	\item
		$|x - a|$ is the distance between $x$ and $a$.
\end{itemize}

Properties of the absolute value
\begin{enumerate}
	\item
		$|x| \geq 0$
	\item
		$|x| = 0 \Leftrightarrow x = 0$
	\item
		$|xy| = |x| \cdot |y|$ \\
		$\left|\dfrac{x}{y}\right| = \dfrac{|x|}{|y|}, y \neq 0$
	\item
		$|x + y| \leq |x| + |y|$ \\
		$|x| - |y| \leq |x - y|$
	\item
		If $a > 0$ then \\
		$|x| < a \Leftrightarrow -a < x < a$ \\
		$|x| > a \Leftrightarrow x > a \text{ or } x < -a$
\end{enumerate}

Other consequences
\begin{enumerate}
	\item
		$|-x| = |x|$
	\item
		If $a > 0$ then \\
		$|x - b| < a \Leftrightarrow b - a < x < b + a$
	\item
		$|x - y| = 0 \Leftrightarrow x = y$
	\item
		$\bigl||x| - |y|\bigr| \leq |x - y|$
\end{enumerate}

Supremum and Infimum \\
Definition. Let $S \subseteq \reals$ and $u, v \in \reals$.
\begin{enumerate}
	\item
		$u$ is an upper bound of $S$ if for all $s \in S$, $s \leq u$
	\item
		$v$ is a lower bound of $S$ if for all $s \in S$, $s \geq v$
\end{enumerate}
Definition.
\begin{enumerate}
	\item
		If $S$ has an upper [lower] bound, then $S$ is said to be bounded above [below].
	\item
		If $S$ is bounded above and below, then $S$ is said to be bounded.
\end{enumerate}
Remark.
\begin{enumerate}
	\item
		$S$ is bounded above if \\
		$(\exists u \in \reals)(\forall s \in S)(s \leq u)$ \\
		$S$ is bounded below if \\
		$(\exists v \in \reals)(\forall s \in S)(s \geq v)$
	\item
		$S$ is bounded \\
		$\Leftrightarrow (\exists u, v \in \reals)(\forall s \in S)(v \leq s \leq u)$ \\
		$\Leftrightarrow (\exists M > 0)(\forall s \in S)(|s| \leq M)$
\end{enumerate}
Definition. Let $S \subseteq \reals$ and $u, v \in \reals$.
\begin{enumerate}
	\item
		$u$ is the supremum (or least upper bound) of $S$ if:
		\begin{enumerate}
			\item
				$u$ is an upper bound of $S$
			\item
				and for all upper bounds $d$ of $S$, $u \leq d$.
		\end{enumerate}
	\item
		$v$ is the infimum (or greatest lower bound) of $S$ if:
		\begin{enumerate}
			\item
				$v$ is a lower bound of $S$
			\item
				and for all lower bounds $b$ of $S$, $v \geq b$.
		\end{enumerate}
\end{enumerate}
Remark.
\begin{enumerate}
	\item
		Notation: \\
		$\sup S = u$ \\
		$\inf S = v$
	\item
		The supremum and infimum of $S$ are not necessarily in $S$.
	\item
		Since $\emptyset$ is bounded above and below by any $a \in \reals$, $\emptyset$ has neither a supremum nor an infimum.
	\item
		$S$ is not bounded above implies that $S$ has no supremum. \\
		$S$ is not bounded below implies that $S$ has no infimum.
\end{enumerate}
Theorem. Let $S \subseteq \reals$. If a supremum [infimum] exists, then it is unique. \\
Proof. Suppose that $u$ and $v$ are suprema of $S$. For the sake of contradiction, assume that $u \neq v$. Without loss of generality, assume that $u < v$. By definition of supremum, $u$ is an upper bound of $S$. Also by definition of supremum, $v \leq d$ for any upper bound $d$ of $S$. Taking $d = u$, we get $v \leq u$. Then $u < v \leq u$, which is absurd. Thus, $u = v$. \\
Theorem. Let $u$ be an upper bound of a non-empty set $S \subseteq \reals$. Then the following are equivalent:
\begin{enumerate}
	\item
		$\sup S = u$
	\item
		$(\forall x \in \reals)(x < u \Rightarrow (\exists s \in S)(x < s))$
	\item
		$(\forall \varepsilon > 0)(\exists s \in S)(u - \varepsilon < s)$
\end{enumerate}

\end{document}

\documentclass[twocolumn]{article}

\usepackage{amsmath}
\usepackage{amssymb}
\usepackage{enumitem}

\setlist{nosep}

\newcommand{\qed}{$\blacksquare$}
\newcommand{\br}{\vspace{\baselineskip}}
\newcommand{\COMPLEMENT}{^\mathsf{c}}
\let\complement\COMPLEMENT
\let\emptyset\varnothing
\let\eps\varepsilon

\newcommand{\naturals}{\mathbb{N}}
\newcommand{\integers}{\mathbb{Z}}
\newcommand{\rationals}{\mathbb{Q}}
\newcommand{\irrationals}{\rationals\complement}
\newcommand{\reals}{\mathbb{R}}
\newcommand{\complex}{\mathbb{C}}

\setlength{\parindent}{0pt}

\usepackage{changepage}
\newcommand{\boxthis}[1]{\fbox{\begin{minipage}{\linewidth}{#1}\end{minipage}}}

\usepackage{etoolbox}
\newcommand{\zerodisplayskips}{%
  \setlength{\abovedisplayskip}{0pt}%
  \setlength{\belowdisplayskip}{0pt}%
  \setlength{\abovedisplayshortskip}{0pt}%
  \setlength{\belowdisplayshortskip}{0pt}}
\appto{\normalsize}{\zerodisplayskips}
\appto{\small}{\zerodisplayskips}
\appto{\footnotesize}{\zerodisplayskips}

\begin{document}

Advanced Calculus I
\begin{itemize}
	\item
		single variable real analysis
\end{itemize}

Real numbers ($\reals$)
\begin{itemize}
	\item
		$(\reals, +, \cdot)$ is a field.
	\item
		$(\reals, \geq)$ is a partially ordered set.
\end{itemize}

Absolute value
\begin{itemize}
	\item
		$|x| = \begin{cases}
			x & x \geq 0 \\
			-x & x < 0 
		\end{cases}$
	\item
		$|x - a|$ is the distance between $x$ and $a$.
\end{itemize}

Properties of the absolute value
\begin{enumerate}
	\item
		$|x| \geq 0$
	\item
		$|x| = 0 \Leftrightarrow x = 0$
	\item
		$|xy| = |x| \cdot |y|$ \\
		$\left|\dfrac{x}{y}\right| = \dfrac{|x|}{|y|}, y \neq 0$
	\item
		$|x + y| \leq |x| + |y|$ \\
		$|x| - |y| \leq |x - y|$
	\item
		If $a > 0$ then \\
		$|x| < a \Leftrightarrow -a < x < a$ \\
		$|x| > a \Leftrightarrow x > a \text{ or } x < -a$
\end{enumerate}

Other consequences
\begin{enumerate}
	\item
		$|-x| = |x|$
	\item
		If $a > 0$ then \\
		$|x - b| < a \Leftrightarrow b - a < x < b + a$
	\item
		$|x - y| = 0 \Leftrightarrow x = y$
	\item
		$\bigl||x| - |y|\bigr| \leq |x - y|$
\end{enumerate}

Supremum and Infimum \\
Definition. Let $S \subseteq \reals$ and $u, v \in \reals$.
\begin{enumerate}
	\item
		$u$ is an upper bound of $S$ if for all $s \in S$, $s \leq u$
	\item
		$v$ is a lower bound of $S$ if for all $s \in S$, $s \geq v$
\end{enumerate}
Definition.
\begin{enumerate}
	\item
		If $S$ has an upper [lower] bound, then $S$ is said to be bounded above [below].
	\item
		If $S$ is bounded above and below, then $S$ is said to be bounded.
\end{enumerate}
Remark.
\begin{enumerate}
	\item
		$S$ is bounded above if \\
		$(\exists u \in \reals)(\forall s \in S)(s \leq u)$ \\
		$S$ is bounded below if \\
		$(\exists v \in \reals)(\forall s \in S)(s \geq v)$
	\item
		$S$ is bounded \\
		$\Leftrightarrow (\exists u, v \in \reals)(\forall s \in S)(v \leq s \leq u)$ \\
		$\Leftrightarrow (\exists M > 0)(\forall s \in S)(|s| \leq M)$
\end{enumerate}
Definition. Let $S \subseteq \reals$ and $u, v \in \reals$.
\begin{enumerate}
	\item
		$u$ is the supremum (or least upper bound) of $S$ if:
		\begin{enumerate}
			\item
				$u$ is an upper bound of $S$
			\item
				and for all upper bounds $d$ of $S$, $u \leq d$.
		\end{enumerate}
	\item
		$v$ is the infimum (or greatest lower bound) of $S$ if:
		\begin{enumerate}
			\item
				$v$ is a lower bound of $S$
			\item
				and for all lower bounds $b$ of $S$, $v \geq b$.
		\end{enumerate}
\end{enumerate}
Remark.
\begin{enumerate}
	\item
		Notation: \\
		$\sup S = u$ \\
		$\inf S = v$
	\item
		The supremum and infimum of $S$ are not necessarily in $S$.
	\item
		Since $\emptyset$ is bounded above and below by any $a \in \reals$, $\emptyset$ has neither a supremum nor an infimum.
	\item
		$S$ is not bounded above implies that $S$ has no supremum. \\
		$S$ is not bounded below implies that $S$ has no infimum.
\end{enumerate}

Theorem. \\
Let $S \subseteq \reals$. If a supremum [infimum] exists, then it is unique. \\
Proof. \\
Suppose that $u$ and $v$ are suprema of $S$. For the sake of contradiction, assume that $u \neq v$. Without loss of generality, assume that $u < v$. By definition of supremum, $u$ is an upper bound of $S$. Also by definition of supremum, $v \leq d$ for any upper bound $d$ of $S$. Taking $d = u$, we get $v \leq u$. Then $u < v \leq u$, which is absurd. Thus, $u = v$. \qed \\

Theorem. \\
Let $u$ be an upper bound of a non-empty set $S \subseteq \reals$. Then the following are equivalent:
\begin{enumerate}
	\item
		$\sup S = u$
	\item
		$(\forall x \in \reals)(x < u \Rightarrow (\exists s \in S)(x < s))$
	\item
		$(\forall \varepsilon > 0)(\exists s \in S)(u - \varepsilon < s)$
\end{enumerate}
Proof. \\
$\left[(1) \Rightarrow (2)\right]$ Suppose that $\sup S = u$. Let $x < u$. Since $u$ is the least upper bound then any number less than $u$ is not an upper bound. In particular, $x$ is not an upper bound. Thus, there exists $s \in S$ such that $x < s$. \\
$\left[(2) \Rightarrow (3)\right]$ Suppose that $(\forall x \in \reals)(x < u \Rightarrow (\exists s \in S)(x < s))$. Let $\varepsilon > 0$. Take $x = u - \varepsilon < u$. Thus, there exists $s \in S$ such that $x < s$ so that $u - \varepsilon < s$. \\
$\left[(3) \Rightarrow (2)\right]$ Suppose that $(\forall \varepsilon > 0)(\exists s \in S)(u - \varepsilon < s)$. For the sake of contradiction, suppose that $u$ is not the supremum of $S$. Thus, there exists $w$ such that $w$ is an upper bound of $S$ and $w < u$. Take $\varepsilon = u - w > 0$. Then, there exists $s$ such that $u - \varepsilon < s$ so that $w < s$. Thus, $w$ is not an upper bound, which is a contradiction. \qed \\

Theorem. \\
Let $v$ be a lower bound of a non-empty set $S \subseteq \reals$. Then the following are equivalent:
\begin{enumerate}
	\item
		$\inf S = v$
	\item
		$(\forall x \in \reals)(x > v \Rightarrow (\exists s \in S)(x > s)$
	\item
		$(\forall \varepsilon > 0)(\exists s \in S)(v + \varepsilon > s)$
\end{enumerate}
Proof. \\
Left as an exercise. \qed \\

Definition. \\
Let $a \in \reals$ and $S \subseteq \reals$.
\begin{enumerate}
	\item
		$a + S = \{a + s \mid s \in S\}$
	\item
		$-S = \{-s \mid s \in S\}$
\end{enumerate} \br

Theorem. \\
Let $S \subseteq \reals$ and $a \in \reals$.
\begin{enumerate}
	\item
		If $S$ is bounded above then $\sup (a + S) = a + \sup S$.
	\item
		If $S$ is bounded below then $\inf (a + S) = a + \inf S$.
	\item
		If $S$ is counded then
		\begin{enumerate}
			\item
				$\inf (-S) = -\sup S$
			\item
				$\sup (-S) = -\inf S$
		\end{enumerate}
\end{enumerate}
Proof. \\
Let $S \subseteq \reals$ and $a \in \reals$.
\begin{enumerate}
	\item
		Let $u = \sup S$ and $v = \sup (a + S)$. Since $u = \sup S$, we know that $u$ is an upper bound of $S$. Hence, for all $s \in S$, $s \leq u$. Thus, for all $s \in S$, $a + s \leq u + a$. Then, $u + a$ is an upper bound of $S$. Therefore, since $v$ is the least upper bound of $S$, $v \leq u + a$. Since $v = \sup (a + S)$, we know that $v$ is an upper bound of $a + S$. Hence, for all $s \in S$, $a + s \leq v$ so that $s \leq v - a$ for all $s \in S$. Thus, $v$ is an upper bound of $S$. Since $u$ is the least upper bound of $S$, we know that $u \leq v - a$ so that $u + a \leq v$.
	\item
		Exercise.
	\item
		Exercise. \qed
\end{enumerate} \br

The Completeness Axiom of $\reals$
\begin{itemize}
	\item
		every non-empty subset of $\reals$ that has an upper [lower] bound has a supremum [infimum].
\end{itemize}

Theorem. Archimedean Property.
\begin{enumerate}
	\item
		For every $x \in \reals$, there exists $n \in \naturals$ such that $x < n$.
	\item
		For every $y > 0$, there exists $n \in \naturals$ such that $\frac{1}{n} < y$.
\end{enumerate}
Proof.
\begin{enumerate}
	\item
		Suppose, for the sake of contradiction, that there exists $x \in \reals$ such that for every $n \in \naturals$, $x \geq n$. Thus, $x$ is an upper bound of $\naturals$. Also, $\naturals$ is clearly non-empty. Thus, by the completeness axiom of $\reals$, there exists $u \in \reals$ such that $u = \sup \naturals$. Since $1 > 0$, by a previous theorem, we know that there exists $n \in \naturals$ such that $u - 1 < n$ so that $u < n + 1 \in \naturals$. Thus, $u$ is not an upper bound of $\naturals$ so that $u$ is not a supremum of $\naturals$, which is a contradiction.
	\item
		Let $y > 0$. Take $x = 1 / y$, then there exists $n \in \naturals$ such that $\frac{1}{y} = x < n$ so that $\frac{1}{n} < y$. \qed
\end{enumerate} \br

Corollary. \\
For any $y > 0$, there exists $n \in \naturals$ such that $n - 1 \leq y < n$. \\
Proof. \\
Let $y > 0$. Consider the set $S = \{n \in \naturals \mid y < n \}$. By the Archimedean Property, $S \neq \emptyset$. By the well-ordering principle, there exists $n \in S$ such that $n = \min S$ so that for all $m \in S$, $m \geq n$. Since $n \in S$, we know that $y < n$. From the fact that $n$ is the least element of $S$, we get that $m < n$ implies that $m \not\in S$ for any $m \in \reals$. In particular, $n - 1 < n$ so that $n - 1 \not\in S$. Thus, $n - 1 \leq y$. \qed \\

Theorem. Density Theorem. \\
For every $x, y \in \reals$ such that $x < y$, there exists $r \in \rationals$ such that $x < r < y$. \\
Proof. \\
Since $y - x > 0$, then by the Archimedean Property, there exists $n \in \naturals$ such that $\frac{1}{n} < y - x$. Thus, $1 < |ny - nx|$. This means that the distance of $ny$ from $nx$ is greater than 1 so that there exists $m \in \integers$ such that $nx < m < ny$ and thus, $x < \frac{m}{n} < y$. Taking $r = \frac{m}{n}$, we are done. \qed \\

Corollary. \\
For every $x, y \in \reals$ such that $x < y$, there exists $r' \in \irrationals$ such that $x < r' < y$. \\
Proof. \\
Exercise. \qed \\

Nested Interval Property. \\

Definition. \\
A set $S \subseteq \reals$ that has at least two elements is an interval if for every $s, r \in S$ such that $r < s$, we have that $\{ x \in \reals \mid r < x < s \} \subseteq S$. \\

Definition. \\
A collection $(I_n)_{n = 1}^\infty$ of intervals in $\reals$ is said to be nested if $I_{j + 1} \subset I_j$ for all $j \in \naturals$. \\

Theorem. Nested Interval Property. \\
Let the collection $([a_n, b_n])_{n = 1}^\infty$ be nested. Then the following are true:
\begin{enumerate}
	\item
		$\bigcap_{n = 1}^\infty [a_n, b_n] \neq \emptyset$
	\item
		$\inf \{ b_n - a_n \mid n \in \naturals \} = 0$ implies $\exists! x \in \reals$ such that $\bigcap_{n = 1}^\infty [a_n, b_n] = \{ x \}$
\end{enumerate}
Proof. \\
Observe that since $([a_n, b_n])$ is nested, then for all $i, j \in \naturals$ such that $i \leq j$, we have that $a_i \leq a_j$ and $b_i \geq b_j$. And since each $[a_n, b_n]$ is an interval, we know that $a_n < b_n$ for all $n \in \naturals$. Hence, for any $n \in \naturals$, we have that $1 \leq n$ so that $b_n \leq b_1$ and thus, $a_n < b_1$.
\begin{enumerate}
	\item
		Let $A = \{ a_n \mid n \in \naturals \}$. Since $A$ is non-empty and bounded above by $b_1$, then by the completeness axiom of $\reals$, there exists $x \in \reals$ such that $\sup A = x$ so that $x$ is an upper bound of $A$. Thus, $a_n \leq x$ for any $n \in \naturals$. Let $m, n \in \reals$. If $m \leq n$ then $a_m \leq a_n < b_n$. If $m > n$ then $a_m < b_m \leq b_n$. Thus, $b_n$ is an upper bound of $A$. Since $x$ is the least upper bound of $A$, we have that $x \leq b_n$.
	\item
		Let $B = \{ b_n \mid n \in \naturals \}$. Similar arguments show that if $y = \inf B$ then for any $m, n \in \naturals$, $a_m \leq y \leq b_n$. Assume $\inf \{ b_n - a_n \mid n \in \naturals \} = 0$. Note that if $z \in \bigcap_{n = 1}^\infty [a_n, b_n]$, then $x \leq z \leq y$. Otherwise, if for example, $z > y$, then there exists $n \in \naturals$ such that $b_n < z$ so that $z \not\in [a_n, b_n]$. This implies that $\bigcap_{n = 1}^\infty [a_n, b_n] \subseteq [x, y]$. Note that for all $n \in \naturals$, $b_n - a_n \geq y - x$ since $b_n \geq y$ and $a_n \leq x$. Thus, $y - x$ is a lower bound of $\{ b_n - a_n \mid n \in \naturals \}$. Since $0 = \inf \{ b_n - a_n \mid n \in \naturals \}$, we have that $y - x \leq 0$ so that $y \leq x$. Since $x \leq y$, then $x = y$. \qed
\end{enumerate} \br

Topology.
\begin{itemize}
	\item
		study of properties of a space that is preserved under continuous deformations
\end{itemize} \br

Definition. \\
Let $\eps > 0$ and $x \in \reals$. The $\eps$-neighborhood of $x$, denoted by $N_\eps(x)$, is the set $N_\eps(x) = \{ y \in \reals \mid |x - y| < \eps \}$, also called the ball about $x$ of radius $\eps$. \\

Definition.
\begin{enumerate}
	\item
		A set $G \subseteq \reals$ is open if for all $x \in G$, there exists $\eps > 0$ such that $N_\eps(x) \subseteq G$.
	\item
		A set $F \subseteq \reals$ is closed if $F\complement$ is open.
\end{enumerate} \br

Theorem.
\begin{enumerate}
	\item
		The arbitrary union of open sets and finite intersections of open sets are open.
	\item
		The arbitrary intersection of closed sets and finite unions of closed sets are closed.
\end{enumerate}
Proof.
\begin{enumerate}
	\item
		Let $G_\alpha$ be open for all $\alpha \in \reals$. Define $G = \bigcup_{\alpha \in \reals} G_\alpha$. Let $x \in G$. Then, there exists $\alpha \in \reals$ such that $x \in G_\alpha$. Since $G_\alpha$ is open, we know that there exists $\eps > 0$ such that $N_\eps(x) \subseteq G_\alpha$. Thus, $N_\eps(x) \subseteq G_\alpha \subseteq G$. \\
		Let $G_1, G_2, \ldots, G_n$ be open sets.  Define $G = \bigcap_{i = 1}^n G_i$. Let $x \in G$. Then, for $i = 1, 2, \ldots, n$, $x \in G_i$. Since $G_i$ is open, there exists $\eps_i > 0$ such that $N_{\eps_i}(x) \subseteq G_i$. Take $\eps = \min \{ \eps_1, \eps_2, \ldots, \eps_n \}$. Then, for $i = 1, 2, \ldots, n$, we have that $N_\eps(x) \subseteq N_{\eps_i}(x)$ so that $N_\eps(x) \subseteq G_i$. Thus, $N_\eps(x) \subseteq \bigcap_{i = 1}^n G_i = G$.
	\item
		Exercise. \qed
\end{enumerate} \br

Definition. \\
Let $x \in \reals$ and $A \subseteq \reals$. We say that $x$ is a cluster point of $A$ if for any $\eps > 0$, we have that $(A \setminus \{ x \}) \cap N_\eps(x) \neq \emptyset$. We denote by $A'$ the set of all cluster points of $A$. \\

Example. Proof is an exercise. \\
\begin{tabular}{l | l}
$A$ & $A'$ \\ \hline
$[a, b]$, $(a, b)$ & $[a, b]$ \\
finite & $\emptyset$ \\
$\naturals$ & $\emptyset$ \\
$\rationals$, $\irrationals$ & $\reals$
\end{tabular} \br

Trivia. \\
$G$ is dense in $F$ if $G' = F$. \\

Proof of $\rationals' = \reals$. \\
Let $x \in \reals$. Let $\eps > 0$. Since $x - \eps < x$, we know by the density theorem that there exists $y \in \rationals$ such that $x - \eps < y < x$ so that $y \neq x$. Thus, $x - y < \eps$ so that $|x - y| < \eps$. \qed \\

Theorem. \\
A set $F \subseteq \reals$ is closed if and only if it contains all of its cluster points. \\
Proof. \\
Let $F \subseteq \reals$ be closed. Let $x \in F'$. Suppose $x \not\in F$. Then, $x \in F\complement$. Note that $F\complement$ is open. Hence, there exists $\eps > 0$ such that $N_\eps(x) \subseteq F\complement$. Thus, no element of $F$ is in $N_\eps(x)$ so that $x$ is not a cluster point, which is a contradiction. \\
Let $F$ contain all of its cluster points. Let $x \in F\complement$. Then, $x$ is not a cluster point of $F$. Thus, there exists $\eps > 0$ such that $(F \setminus \{ x \}) \cap N_\eps(x) = \emptyset$. Since $x \not\in F$, we get that $F \setminus \{ x \} = F$ so that $F \cap N_\eps(x) = \emptyset$. Therefore, $N_\eps(x) \subseteq F\complement$. \qed \\

Definition. \\
The closure of $A$, denoted by $\bar{A}$, is the smallest closed set containing $A$. We state the following facts without proof:
\begin{enumerate}
	\item
		$\bar{A} = A \cup A'$
	\item
		$A$ is closed if and only if $\bar{A} = A$
\end{enumerate} \br

Compact Sets. \\

Definition. \\
Let $A \subseteq \reals$. An open cover of $A$ is a collection $\mathcal{G} = \{ G_\alpha \}_{\alpha \in \reals}$ of open setssuch that $A \subseteq \bigcup_{\alpha \in \reals} G_\alpha$. \\
A subcover of $\mathcal{G}$ is a subcollection $\{ G'_\alpha \}_{\alpha \in \reals}$ such that $A \subseteq \bigcup_{\alpha \in \reals} G'_\alpha$. \\

Definition. \\
A set $K \subseteq \reals$ is compact if every open cover of $K$ has a finite subcover. \\

Lemma. \\
Let $n \in \naturals$. Then $n \leq 2^n$. \\
Proof. \\
Let $n \in \naturals$. If $n = 1$, then $n = 1 < 2 = 2^1 = 2^n$. Suppose that $n \leq 2^n$. Then $n + 1 \leq n + n = 2n \leq 2(2^n) = 2^{n + 1}$. Thus, by the principle of mathematical induction, $n \leq 2^n$ for any $n \in \naturals$. \\

Theorem. \\
$[a, b]$ is compact. \\
Proof. \\
Suppose, for the sake of contradiction, that $\mathcal{G} = \{ G_\alpha \}$ is an open cover of $[a, b]$ that does not have a finite subcover. \\
Consider the intervals $[a, \frac{a + b}{2}]$ and $[\frac{a + b}{2}, b]$. Since $[a, b]$ does not have a finite subcover, then one of the two is also not finitely covered, say $I_1$. \\
Bisecting $I_1$, we again can conclude that one of these two subintervals will not be finitely covered, say $I_2$. \\
Continuing this, we obtain a sequence of intervals $\{ I_n \}_{n = 1}^\infty$ such that
$$I_1 \supseteq I_2 \supseteq I_3 \supseteq \cdots \text{.}$$
Clearly, the sequence is a nested set of closed and bounded intervals. \\
Suppose that $I_n = [a_n, b_n]$. If $n = 1$, then $b_1 - a_1 = \frac{b - a}{2}$. If $b_n - a_n = \frac{b - a}{2^n}$ then $b_{n + 1} - a_{n + 1} = \frac{1}{2}(b_n - a_n)$. So by the principle of mathematical induction, $b_n - a_n = \frac{b - a}{2^n}$ for any $n \in \naturals$. \\
Consider the set $A = \{ b_n - a_n \mid n \in \naturals \} = \{ \frac{b - a}{2^n} \mid n \in \naturals \}$. Note that $\frac{b - a}{2^n} > 0$ for any $n \in \naturals$ so that $0$ is a lower bound of $A$. Let $\eps > 0$. By the archimedean property, there exists $r \in \naturals$ such that $\frac{1}{r} < \eps$. Also by the archimedean property, there exists $s \in \naturals$ such that $r(b - a) < s$ so that $\frac{b - a}{s} < \frac{1}{r}$. Taking $n = s$, we get $\frac{b - a}{2^n} \leq \frac{b - a}{s} < \frac{1}{r} < \eps$. Therefore, $\inf A = 0$. \\
By the nested interval property, $\exists! x \in \reals$ such that $\bigcap_{n = 1}^\infty I_n = \{ x \}$. Since $x \in [a, b]$, there exists $\alpha$ such that $x \in G_\alpha$. Since $G_\alpha$ is open, there exists $\eps$ such that $N_\eps(x) \subseteq G$. Choose $N$ large enough such that $\frac{b - a}{2^N} < \eps$. Then, $I_N \subseteq N_\eps(x)$ so that $I_N$ is finitely covered, which is a contradiction. \qed \\

Theorem. Heine Borel Theorem. \\
A set $K \subseteq \reals$ is compact if and only if it is closed and bounded. \\
Proof. \\
Suppose $K$ is compact. \\
Consider $\mathcal{G} = \{ G_n \}$ where $G_n = (-n, n)$. Clearly, $\mathcal{G}$ covers $\reals$. In particular, it covers $K$. Since $K$ is compact, there exists a finite subcover $\mathcal{G'} = \{ G_{m_1}, G_{m_2}, \cdots, G_{m_n} \}$. Take $M = \max \{ m_1, m_2, \cdots, m_n \}$. Thus, $\bigcup_{i = 1}^n G_{m_i} = (-M, M) \supseteq K$, so that $K$ is bounded. \\
Let $x \in K\complement$. Take $\mathcal{G} = \{ G_n \}$, where $G_n = \{ y \mid |x - y| > \frac{1}{n} \}$. Then, $\bigcup_{n = 1}^\infty G_n = \reals \setminus \{ x \} \supseteq K$. Since $K$ is compact, there exists a finite subcover $\mathcal{G'} = \{ G_{m_1}, G_{m_2}, \cdots, G_{m_n} \}$. Take $M = \max \{ m_1, m_2, \cdots, m_n \}$. Thus, $\bigcup_{i = 1}^n G_{m_i} = G_M \supseteq K$. Take $\eps = \frac{1}{M}$ so that $N_\eps(x) \subseteq G_M\complement$ and thus, $N_\eps(x) \subseteq K\complement$. \\
Suppose $K$ is closed and bounded. Let $\mathcal{G} = \{ G_\alpha \}$ be an open cover for $K$. Since $K$ is bounded, then there exists $M > 0$ such that $K \subseteq [-M, M]$. Since $K$ is closed, $K\complement$ is open. Note that $\bigcup G_\alpha \supseteq K$ so that $\bigcup G_\alpha \cup K\complement \supseteq K \cup K\complement = \reals$. Thus, $\mathcal{G} \cup \{ K\complement \}$ is an open cover for $\reals$ and in particular, $[-M, M]$. Since $[-M, M]$ is compact, there exists a finite subcover of $\mathcal{G} \cup \{ K\complement \}$ so that there exists $\mathcal{G'} = \{ G_{m_1}, G_{m_2}, \cdots, G_{m_n} \}$ such that $\bigcup_{i = 1}^n G_{m_i} \cup K\complement \supseteq [-M, M] \supseteq K$. Thus, $\bigcup_{i = 1}^n G_{m_i} \supseteq K$. \qed \\

Sequences in $\reals$ \\

Definition. \\
A sequence in $\reals$ is a function defined on the set $\naturals$ and whose range is contained in $\reals$. \\

Definition. \\
Let $(x_n)$ be a sequence of real numbers. We say that $(x_n)$ approaches a real number $x$, denoted by $\lim x_n = x$ if and only if for every $\eps > 0$, there exists $N \in \naturals$ such that if $n \in N$ and $n \geq N$, $|x_n - x| < \eps$. In this case, we say that $(x_n)$ is convergent. Otherwise, we say that $(x_n)$ is divergent. \\

Remark.
\begin{enumerate}
	\item
	$\lim x_n = x$ \\
	$\Leftrightarrow$ $\forall \eps > 0$, $\exists N \in \naturals$ such that if $n \geq N$, then
	$$|x_n - x| < \eps$$
	$\Leftrightarrow$ $\forall \eps > 0$, $\exists N \in \naturals$ such that $\forall n \geq N$,
	$$|x_n - x| < \eps$$
	\item
	$(x_n)$ is convergent $\Leftrightarrow$ $\exists x \in \reals$ such that $\lim x_n = x$. \\
	$(x_n)$ is divergent $\Leftrightarrow$ $\forall x \in \reals$, $\exists \eps > 0$ such that $\forall N \in \naturals$, $\exists n \geq N$ such that $|x_n - x| \geq \eps$.
	\item
	$\lim x_n = x$ implies that there are only a finite number of terms outside of $N_\eps(x)$.
	\item
	If $N$ satisfies our definition, then any $N_0 > N$ will also satisfy our definition.
\end{enumerate} \br

Example.
\begin{enumerate}
	\item
	$\lim c = c$. \\
	Proof. \\
	Let $\eps > 0$. Choose $N = 1$. Thus if $n \geq N$ then $|x_n - x| = |c - c| = 0 < \eps$.
	\item
	$\lim \frac{1}{n} = 0$. \\
	Proof. \\
	Let $\eps > 0$. By the Archimedean property, there exists $N \in \naturals$ such that $\frac{1}{N} < \eps$. Thus if $n \geq N$ then $|x_n - x| = |\frac{1}{n} - 0| = \frac{1}{n} \leq \frac{1}{N} < \eps$.
	\item
	$\lim \frac{n^2 - 5}{n^2 + 5} = 1$. \\
	Proof. \\
	Let $\eps > 0$. By the Archimedean property, there exists $N \in \naturals$ such that $\frac{1}{N} < \frac{\eps}{10}$. Thus if $n \geq N$ then $|x_n - x| = |\frac{n^2 - 5}{n^2 + 5} - 1| = |\frac{-10}{n^2 + 5}| = \frac{10}{n^2 + 5} < \frac{10}{n^2} \leq \frac{10}{n} \leq \frac{10}{N} < \eps$.
	\item
	Let $\eps > 0$. By the Archimedean property, there exists $N \in \naturals$ such that $\frac{10}{\eps} + 5 < N$. Thus if $n \geq N$ then $|x_n - x| = |\frac{n^2 + 5}{n^2 - 5} - 1| = |\frac{10}{n^2 - 5}| = \frac{10}{n^2 - 5} \leq \frac{10}{N - 5} < \eps$.
\end{enumerate} \br

Theorem. \\
If for all $\eps > 0$, $0 \leq x < \eps$, then $x = 0$. \\
Proof. \\
Let $\eps > 0$. Suppose that $0 \leq x < \eps$. Then $0 \leq x$. Suppose, for the sake of contradiction, that $x \neq 0$. Then $0 < x$. Take $\eps = x$. Then $0 \leq x < x$ so that $x < x$, which is a contradiction. Thus, $x = 0$. \qed \\

Corollary. \\
If $|x - y| < \eps$ for all $\eps > 0$, then $x = y$. \\
Proof. \\
Exercise. \qed \\

Theorem. \\
If $\lim x_n$ exists, then it is unique. \\
Proof. \\
Suppose that $x$ and $y$ are limits of $x_n$. Let $\eps > 0$. Since $\lim x_n = x$, there exists $N_1 \in \naturals$ such that $|x_n - x| < \frac{\eps}{2}$ for all $n \geq N_1$. Since $\lim x_n = y$, there exists $N_2 \in \naturals$ such that $|x_n - y| < \frac{\eps}{2}$ for all $n \geq N_2$. Take $N = \max \{ N_1, N_2 \}$. Then $|x_N - x| < \frac{\eps}{2}$ so that $-\frac{\eps}{2} < x_N - x < \frac{\eps}{2}$. Also, $|y - x_N| = |x_N - y| < \frac{\eps}{2}$ so that $-\frac{\eps}{2} < y - x_N < \frac{\eps}{2}$. Hence, $-\eps < x_N - x + y - x_N < \eps$ so that $|y - x| < \eps$. Thus, by the previous corollary, $x = y$. \qed \\

Theorem. \\
If $(x_n)$ is convergent then $\{ x_n \mid n \in \naturals \}$ is bounded. \\
Proof. \\
Let $\lim x_n = x$. Then there exists $N \in \naturals$ such that $|x_n - x| < 1$ for all $n \geq N$. Then $|x_n| - |x| \leq |x_n - x| < 1$ so that $|x_n| < 1 + |x|$ for all $n \geq N$. Take $M = \max \{ 1 + |x|, |x_1|, |x_2|, \ldots, |x_{N - 1} \}$. Then $|x_n| \leq M$ for all $n \in \naturals$, so that $\{ x_n \mid n \in \naturals \}$ is bounded. \qed \\

Definition. \\
Let $(x_n)$ and $(y_n)$ be sequences.
\begin{enumerate}
	\item
	$(x_n) + (y_n) := (x_n + y_n)$
	\item
	$(x_n)(y_n) := (x_ny_n)$
	\item
	$\frac{(x_n)}{(y_n)} := (\frac{x_n}{y_n})$, $y_n \neq 0$
\end{enumerate} \br

Theorem. \\
Let $(x_n)$ and $(y_n)$ be convergent sequences with $\lim x_n = x$ and $\lim y_n = y$. Then the following are true.
\begin{enumerate}
	\item
	$\lim x_n + y_n = x + y$
	\item
	$\lim x_ny_n = xy$
	\item
	$\lim \frac{1}{x_n} = \frac{1}{x}$, provided that $x \neq 0$ and $x_n \neq 0$ $\forall n \in \naturals$
\end{enumerate}
Proof.
\begin{enumerate}
	\item
	Let $\eps > 0$. Since $\lim x_n = x$, there exists $N_1 \in \naturals$ such $|x_n - x| < \frac{\eps}{2}$ for any $n \geq N_1$. Since $\lim y_n = y$, there exists $N_2 \in \naturals$ such that $|y_n - y| < \frac{\eps}{2}$ for any $n \geq N_2$. Take $N = \max \{ N_1, N_2 \}$. Thus if $n \geq N$, then $|(x_n + y_n) - (x + y)| = |(x_n - x) + (y_n - 2)| \leq |x_n - x| + |y_n - y| < \frac{\eps}{2} + \frac{\eps}{2} = \eps$.
	\item
	Let $\eps > 0$. Since $(y_n)$ is convergent, there exists $M > 0$ such that $|y_n| < M$ for all $n \in \naturals$. Take $K = \max \{ M, |x| \}$. Since $\lim x_n = x$, then there exists $N_1 \in \naturals$ such that $|x_n - x| < \frac{\eps}{2K}$ for any $n \geq N_1$. Since $\lim y_n = y$, there exists $N_2 \in \naturals$ such that $|y_n - y| < \frac{\eps}{2K}$ for any $n \geq N_2$. Take $N = \max \{ N_1, N_2 \}$. Thus if $n \geq N$, then $|x_ny_n - xy| = |(x_ny_n - xy_n) + (xy_n - xy)| \leq |y_n(x_n - x)| + |x(y_n - y)| = |y_n| \cdot |x_n - x| + |x| \cdot |y_n - y| < K|x_n - x| + K|y_n - y| < \frac{\eps}{2} + \frac{\eps}{2} = \eps$.
	\item
	Let $\eps > 0$. Since $\lim x_n = x$ and $x \neq 0$, there exists $N_1 \in \naturals$ such that $|x_n - x| < \frac{|x|}{2}$ for any $n \geq N_1$. Hence, $|x| - |x_n| < \frac{|x|}{2}$ so that $\frac{|x|}{2} < |x_n|$. And since $\lim x_n = x$, there exists $N_2 \in \naturals$ such that $|x_n - x| < \frac{|x|^2}{2}$ for any $n \geq N_2$. Take $N = \max \{ N_1, N_2 \}$. Thus if $n \geq N$, then $|\frac{1}{x_n} - \frac{1}{x}| = |\frac{x - x_n}{x_nx}| = \frac{|x_n - x|}{|x_n| \cdot |x|} < \frac{\frac{x}{2}^2 \cdot \eps}{\frac{|x|}{2} \cdot |x|} = \eps$. \qed
\end{enumerate} \br

Theorem. \\
If $(x_n)$ diverges and $(y_n)$ converges then $(x_n + y_n)$ diverges. \\
Proof. \\
Suppose, for the sake of contradiction, that $(x_n + y_n)$ converges. Consider $(z_n) := (x_n + y_n) + (-1)(y_n)$, then this sequence converges by the previous theorem. Since $(z_n) = (y_n)$, we have a contradiction. \qed \\

Theorem. \\
Let $(x_n)$ and $(y_n)$ be convergent sequences. Then the following are true.
\begin{enumerate}
	\item
	If $x_n \geq 0$ for any $n \in \naturals$, then $\lim x_n \geq 0$.
	\item
	If $x_n \geq y_n$ for any $n \in \naturals$, then $\lim x_n \geq \lim y_n$.
\end{enumerate}
Proof.
\begin{enumerate}
	\item
	Let $\lim x_n = x$. Suppose, for the sake of contradiction, that $x < 0$. Then $-x > 0$, so that there exists $N \in \naturals$ such that $|x_n - x| < -x$ for any $n \geq N$. In particular, $|x_N - x| < -x$. But $|x_N - x| = x_N - x$, so that $x_N - x < -x$. Hence, $x_N < 0$, which is a contradiction. Thus, $x \geq 0$.
	\item
	Consider $(z_n) = (x_n) + (-1)(y_n)$. By a previous theorem, $(z_n)$ is convergent. Also, $z_n = x_n - y_n \geq 0$ for any $n \in \naturals$ so that $\lim z_n \geq 0$. But $\lim z_n = \lim x_n + -1 \cdot \lim y_n = \lim x_n - \lim y_n$. Hence, $\lim x_n - \lim y_n \geq 0$ so that $\lim x_n \geq \lim y_n$. \qed
\end{enumerate} \br

Theorem. Squeeze Theorem for Limits. \\
If $(x_n)$, $(y_n)$, and $(z_n)$ are sequences satisfying
\begin{enumerate}
	\item
	$x_n \leq y_n \leq z_n$ for any $n \in \naturals$ and
	\item
	$(x_n)$ and $(z_n)$ are convergent with $\lim x_n = x$ and $\lim z_n = x$,
\end{enumerate}
then $\lim y_n = x$. \\
Proof. \\
Let $\eps > 0$. Since $\lim x_n = x$, there exists $N_1 \in \naturals$ such that if $n \geq N_1$,
$$|x_n - x| < \eps$$
$$\Leftrightarrow -\eps < x_n - x < \eps \text{.}$$
Since $\lim z_n = x$, there exists $N_2 \in \naturals$ such that if $n \geq N_2$,
$$|z_n - x| < \eps$$
$$\Leftrightarrow -\eps < z_n - x < \eps \text{.}$$
Take $N = \max \{ N_1, N_2 \}$. Thus if $n \geq N$, then $-\eps < x_n - x \leq y_n - x \leq z_n - x < \eps$ so that $|y_n - x| < \eps$. \qed \\

Remark. \\
In general, we can assume that $\exists N \in \naturals$ such that $x_n \leq y_n \leq z_n$ $\forall n \geq N$. \\

Theorem. Monotone Subsequence Theorem. \\
Every sequence has a monotone subsequence. \\
Proof. \\
Let $(x_n)$ be a subsequence. Let $S$ be the set of peaks of $(x_n)$, $S := \{ x_m \mid m \leq n \Rightarrow x_m \geq x_n \}$. \\
Suppose $S$ is infinite. Since for all $m_n$ such that $x_{m_n} \in S$, then $x_{m_{n_1}} \geq x_{m_{n_2}}$ if $m_{n_1} \geq m_{n_2}$. Thus the subsequence $(x_{m_n})$ is decreasing. \\
Now, suppose that $S$ is finite. Thus, there exists $N_1 \in \naturals$ such that $N_1 > m$ for any $x_m \in S$. So $x_{N_1}$ is not a peak, so that there exists $N_2 > N_1$ such that $x_{N_1} < x_{N_2}$. So $x_{N_2}$ is not a peak so that there exists $N_3$ such that $x_{N_2} < x_{N_3}$. Continuing this, we obtain a strictly increasing subseqence $(x_{N_k})$. \qed \\

Theorem. Bolzano-Weierstrass Theorem. \\
Every bounded sequence has a convergent subsequence. \\
Proof. \\
Let $(x_n)$ be a bounded sequence. By the monotone subsequence theorem, there exists $(x_{n_k})$ monotone subsequence of $(x_n)$. Sinc $(x_n)$ is bounded, then $(x_{n_k})$ is also bounded. By the monotone convergence theorem, $(x_{n_k})$ is convergent. \qed \\

Theorem. \\
Let $(x_n)$ be a bounded sequence. If every convergent subsequence of $(x_n)$ converges to $x$, then $(x_n)$ converges to $x$. \\
Proof. \\
Suppose, for the sake of contradiction, that $(x_n)$ does not converge to $x$. Then there exists $\eps > 0$ such that for any $N \in \naturals$, there exists $n \geq N$ such that $|x_n - x| \geq \eps$. \\
\boxthis{
Claim. \\
There exists a subsequence $(x_{n_k})$ such that
$$|x_{n_k} - x| \geq \eps \text{.}$$
Proof. \\
Indeed, if $N = 1$, there exists $x_n$ such that
$$|x_{n_1} - x| \geq \eps \text{.}$$
Also, if $N = n_1 + 1$, there exists $x_{n_2}$ such that
$$|x_{n_2} - x| \geq \eps \text{.}$$
Continuing this for $N = n_k + 1$, there exists $X_{n_{k + 1}}$ such that
$$|x_{k + 1} - x| \geq \eps \text{.}$$
And we are done.
}
Since $(x_n)$ is bounded, then $(x_{n_k})$ is bounded. Thus by the Bolzano-Weierstrass theorem, there is a convergent subsequence of $(x_{n_k})$, say $(x_{n_k}')$. By assumption, $\lim x_{n_k}' = x$. Thus there exists $N \in \naturals$ such that if $n_k \geq N$ then $|x_{n_k}' - x| < \eps$. This is a contradiction since $|x_{n_k}' - x| \geq \eps$ by construction of $(x_{n_k})$. \qed \\

Definition. \\
Let $(x_n)$ be a sequence. We say that $y$ is a subsequential limit of $(x_n)$ if there exists a subsequence $(x_{n_k})$ that converges to $y$. \\

Remark.
\begin{enumerate}
	\item
	The subsequence $(x_n)$ will diverge if it has at least two distinct subsequential limits.
	\item
	Sometimes called a limit point, but definitely not a cluster point.
	\item
	$y$ is a subsequential limit of $(x_n)$ if and only if for any $\eps > 0$, $N_\eps(y)$ has infinitely many terms of $(x_n)$.
\end{enumerate} \br

Definition. \\
Let $(x_n)$ be a bounded sequence. Let $a_n = \inf \{ x_m \mid m \geq n \}$ and $b_n = \sup \{ x_m \mid m \geq n \}$.
\begin{enumerate}
	\item
	The limit inferior of $(x_n)$, denoted by $\liminf$, is given by $\lim a_n$.
	\item
	The limit superior of $(x_n)$, denoted by $\limsup$, is given by $\lim b_n$.
\end{enumerate} \br

Remark. \\
\begin{enumerate}
	\item
	The sequences $(a_n)$ and $(b_n)$ are monotone. In particular, $(a_n)$ is increasing and $(b_n)$ is decreasing.
	\item
	If $(x_n)$ is bounded, the $\liminf$ and $\limsup$ will always exist.
\end{enumerate} \br

Example. \\

\end{document}
